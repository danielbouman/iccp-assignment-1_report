\section{Conclusion}

In general, the results for the physical quantities during the simulations were in good agreement with expectation, previous simulated results or experimental results. Especially the correlation functions found were in excellent agreement with expectation and simulated results. However, significant differences were found for the diffusion constant and the specific heat. During simulations the instantaneous specific heat and diffusion constant were calculated, instead of averaging over a preselected time period to approximate the expectation value of some quantities required in the calculations. Followup simulations should investigate the effect on the accuracy if the diffusion constant and the specific heat are calculated over longer time periods. Furthermore, the relative fluctuations in the pressure increased dramatically for lower densities for liquid argon. This could be attributed to the fact that only 108 particles are used for the simulations, so that at low densities the fluctuations in the virial become significant. More particles could be introduced while keeping the simulation times reasonable by several methods\cite{PhysRev.159.98}\cite{thompson1983use}. A significant increase in the accuracy of the physical quantities besides the pressure is not expected when the amount of particles is increased, however.