\section{Equilibrium}
When the simulation starts, the system is far from equilibrium. [[why?]] To push the system towards equilibrium, the particle velocities are rescaled with scaling factor $\lambda$:
\begin{gather}\label{eq:rescale}
    \lambda=\sqrt{\frac{(N-1)3k_BT_D}{\sum_{i=1}^{N} mv_i^{2}}}.
\end{gather}. After each rescaling a jump in the system temperature $T$ towards the desired temperature $T_d$ occurs. This can be seen in figure \ref{fig:temp}, where the system temperature vs time is plotted when the system is still in the equilibration phase. Here, rescaled is performed every 40 time steps. This can be seen as discontinuities in the temperature. 



A simulation for 108 particles was run, figure %\ref{fig:instant_temp}
shows the instantaneous temperature as a function of time steps. The desired temperature was set to 300~K. The temperature fluctuates a lot at the start of the simulation, since the system is far from equilibrium. After around 1000 time steps the system appears to reach equilibrium and the temperature fluctuates around $T_D$. During the equilibration phase the particle velocity is rescaled at intervals of 200 time steps, with the rescaling factor from Eq. \ref{eq:rescale}. Now that the system is in equilibrium, physical quantities can be extracted from the simulation.