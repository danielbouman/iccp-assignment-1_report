\section{Results and discussion}


\subsection{Correlation function}

The simulated correlation function for liquid argon corresponds well with previous found results \cite{PhysRev.136.A405}\cite{yoon1981radial}. The simulated correlation function  for solid argon also corresponds well \cite{franchetti1975radial}\cite{jos}. These correlation functions admit an easy interpretation. The correlation function peaks around $r = 2^{1/6} \sigma$, since there the used Lennard-Jones potential has its minimum. The correlation functions rapidly decays for $r$ smaller than $2^{1/6} \sigma$, since there the repulsive part of the Lennard-Jones potential begins to dominate heavily over the attractive part. The differences between the phases are the most physically relevant. For $r$ greater than $2^{1/6}\sigma$ regularly spaced peaks with high visibility can be seen in the correlation function for solid argon. This corresponds to the particles having a well-defined fixed position relative to each other, which corresponds to an almost lattice-like configuration. This is what we would expect in a solid. For a liquid the peaks are more smeared out. This corresponds to the particles having a less well-defined position, yet the particles are concentrated primarily in each others potential minimum.

% Simulations to run:

% liquid:\\
% rho     T_d\\     
% 0.88    1\\
% 0.8     1\\
% 0.7     1\\

% gas:\\
% rho     T_d\\
% 0.3     3

% solid:\\
% rho     T_d\\
% 1.2     0.5 
Table \ref{table:phys_quan} shows these averaged physical quantities. Table \ref{table:variance} shows the variance of the physical quantities.
\begin{center}
\captionof{table}{Average physical quantities, in equilibrium phase.}
\begin{tabular}{lllllll}
\hline \hline
$\rho(1/\sigma^3)$ & $T_d$ & $T$ & $\beta P/\rho$ & $U(\epsilon)$ & $C_v$ \\
\hline
0.3 & 3.0 & 3.030 & 1.14 & -1.721 & 1.61 \\
0.7 & 1.0 & 0.976 & 0.197 & -4.720 & 1.95 \\
0.8 & 1.0 & 1.023 & 1.841 & -5.274 & 2.11 \\
0.88& 1.0 & 0.967 & 3.258 & -5.756 & 2.13 \\
1.06& 0.827&0.848 & 7.131 & -6.976 & 3.68 \\
1.2 & 0.5 & 0.501 & 27.86 & -6.923 & 4.85 \\
\hline \hline
\end{tabular}
\vspace{-0.1cm}
\captionof*{table}{$T_d$ is the desired temperature, while $T$ is the measured temperature. The density $\rho$ is defined as $N/V$. $U$ is the potential energy and $C_v$ the specific heat.}
\label{table:phys_quan}
\end{center}

\begin{center}
\captionof{table}{Variance of physical quantities, in equilibrium phase.}
\begin{tabular}{lllllll}
\hline \hline
 $\delta T$ & $\delta (\beta P/\rho)$ & $\delta U(\epsilon)$ & $\delta C_v$ \\
\hline
  &  &  &  \\
  &  &  &  \\
  &  &  &  \\
  &  &  &  \\
  &  &  &  \\
  &  &  &  \\
\hline \hline
\end{tabular}
\vspace{-0.1cm}
\captionof*{table}{[[caption]]}
\label{table:variance}
\end{center}

[[ diffusion variance too large to give reliable results ]]
