\section{Results and discussion}


\subsection{Correlation function}

The simulated correlation function for liquid argon corresponds well with previous found results \cite{PhysRev.136.A405}\cite{yoon1981radial}. The simulated correlation function  for solid argon also corresponds well \cite{franchetti1975radial}\cite{jos}. These correlation functions admit an easy interpretation. The correlation function peaks around $r = 2^{1/6} \sigma$, since there the used Lennard-Jones potential has its minimum. The correlation functions rapidly decays for $r$ smaller than $2^{1/6} \sigma$, since there the repulsive part of the Lennard-Jones potential begins to dominate heavily over the attractive part. The differences between the phases are the most physically relevant. For $r$ greater than $2^{1/6}\sigma$ regularly spaced peaks with high visibility can be seen in the correlation function for solid argon. This corresponds to the particles having a well-defined fixed position relative to each other, which corresponds to an almost lattice-like configuration. This is what we would expect in a solid. For a liquid the peaks are more smeared out. This corresponds to the particles having a less well-defined position, yet the particles are concentrated primarily in each others potential minimum.

% Simulations to run:

% liquid:\\
% rho     T_d\\     
% 0.88    1\\
% 0.8     1\\
% 0.7     1\\

% gas:\\
% rho     T_d\\
% 0.3     3

% solid:\\
% rho     T_d\\
% 1.2     0.5    